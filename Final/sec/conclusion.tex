\section{CONCLUSION AND RECOMMENDATIONS}
In conclusion, the development of "Baal Arjun," a 2D arrow shooting game, an OOP in C++ project using the SFML library, has been a rewarding journey for our team. This project has not only resulted in a functional game but has also significantly enriched our collaborative and teamwork skills. The process of designing, coding, and refining the game has brought us together and taught us the importance of effective communication and coordinated effort in achieving a shared goal.

Throughout this project, we gained valuable insights into Object-Oriented Programming principles, honed our proficiency in the C++ programming language, and deepened our understanding of game development through the practical application of SFML. Beyond the technical aspects, we also acquired proficiency in utilizing essential industry tools such as GitHub for version control and Lucidchart for visual documentation, equipping us with practical skills highly relevant to the software development field.

As we move forward, we recommend that future enhancements for "Baal Arjun" include the incorporation of diverse arrow types, improved graphics, additional difficulty levels, personalized avatars, health progression mechanisms, a rewarding system for player achievements, shield functionalities, and a more immersive storytelling experience. These advancements will not only enhance the gameplay but also provide opportunities for continued learning and growth.

In essence, the "Baal Arjun" project has provided us with a comprehensive understanding of software development principles, effective collaboration, and utilization of industry tools, which will undoubtedly serve as a solid foundation for our future endeavors in the field of technology and game development.

\newpage
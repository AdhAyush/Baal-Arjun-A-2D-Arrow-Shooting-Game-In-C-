	\newpage
\section{METHODOLOGY}
To complete our project, we will follow the following methods:

\begin{enumerate}
	\item Research and Planning:
		\begin{enumerate}
			\item  Conduct research on various aspects of game development, including features, fundamental requirements, and tools required.
			\item Familiarize ourselves with the necessary libraries and divide the work among the team members.
			\item Develop comprehensive coding guidelines to ensure self-documenting code.
			
		\end{enumerate}
	\item Algorithm Design: 
		\begin{enumerate}
			
			\item Design the algorithm and create a flowchart based on the gathered rules and information. 
			\item Develop a basic prototype for testing purposes, using the console.

			
		\end{enumerate}
	
	\item  Sources:
		Utilize the following sources as references to support our project:
		\begin{enumerate}
			
			\item Documentation for SFML.
			\item Concepts of object-oriented programming (OOP) and C++ programming from college courses, online references, books, etc.
			
			
		\end{enumerate}
		
	\item Game Design:
	
		\begin{enumerate}
			
			\item Implement the code using an object-oriented programming (OOP) paradigm. 
			\item Use Simple and Fast Multimedia Library to create build the game.
			\item Choose Visual Studio as our Integrated Development Environment (IDE) and visual c++ as the compiler in the Windows Operating System.
			\item Manage our code using GitHub for version control.
			
		\end{enumerate}
	
	\item Maintenance:
	Regularly maintain the project by following these procedures:
		\begin{enumerate}
			
			\item Corrective maintenance:
				\begin{itemize}
					\item Compile and check specific parts of the code for potential bugs and errors, fixing them as necessary.
				\end{itemize}
			
			\item  Preventive maintenance:
				\begin{itemize}
					
					\item Identify errors and take measures to avoid them during runtime and the development cycle.
					\item Comment and document the source code thoroughly and back it up to prevent potential data loss.
					
					
				\end{itemize}
		\end{enumerate}
	\pagebreak
	\item Testing and Debugging:
		\begin{enumerate}
			
			\item Develop a minimum viable product sample for testing and debugging purposes.
			\item Conduct frequent testing and debugging to add features and make necessary edits to the project code based on our requirements and capabilities.
			
		\end{enumerate}
	\item Documentation:
		\begin{enumerate}
			\item Ensure comprehensive documentation of the app, including the addition of copyright licenses.
			\item Ship the completed app with the accompanying documentation.
		\end{enumerate}
\end{enumerate}




